\documentclass{article}

\usepackage[english]{babel}

\usepackage[letterpaper,top=1in,bottom=1in,left=1in,right=1in,marginparwidth=1in]{geometry}

% Useful packages
\usepackage{amsmath}
\usepackage{graphicx}
\usepackage[colorlinks=true, allcolors=blue]{hyperref}

\title{Airline Booking System}
\author{Jean-Denis de Beauvoir, Miguel Canelo, Brian Ortiz}

\begin{document}
\maketitle

\begin{abstract}
The Airline Management System enables efficient handling of passenger, flight, and booking data while ensuring smooth operational throughput even during periods of high demand and numerous concurrent transactions. In addition to managing core functionalities such as flight scheduling, ticketing, and payment processing, the system also supports report generation to aid in various airline operations and decision-making processes. The system integrates a variety of automated modules to streamline workflows across departments, including reservations, customer service, and flight operations. To illustrate its functionality and design, we developed multiple visual models including context flow diagrams, data flow diagrams(Levels 1 and 2), class diagrams, entity-relationship diagrams, and use case diagrams. These models will provide a comprehensive overview of the system's architecture and behavior.
\end{abstract}

\section{Introduction}

Add Text

\section{Context Flow Diagram}

\subsection{Definitions}

\begin{itemize}
\item Employee:
    \begin{itemize}
    \item Those who work for the Airline.
    \end{itemize}
\item Administration:
    \begin{itemize}
    \item Those who run the Airline Management System.
    \end{itemize}
\item Customer:
    \begin{itemize}
    \item The User/Person who will be buying tickets/flights.
    \end{itemize}
\item Aircraft:
    \begin{itemize}
    \item The Plane used to transport different flights/Passengers.
    \end{itemize}
\end{itemize}

\subsection{Symbols}
\begin{itemize}
\item Squares:
    \begin{itemize}
        \item The different squares represent external entities in the context flow diagram.
    \end{itemize}
\end{itemize}

\begin{itemize}
\item Arrows:
    \begin{itemize}
        \item Represent different processes relating entities together and describe what they may do while interacting.
    \end{itemize}
\end{itemize}

\begin{itemize}
\item Circle:
    \begin{itemize}
        \item Represents the Central Entity that all the other external entities interact with.
    \end{itemize}
\end{itemize}

\section{Entity Relationship Diagram}
\subsection{Diagram Symbols}
\begin{itemize}
\item Key Descriptions:
    \begin{itemize}
    \item Primary Key
    \begin{itemize}
        \item Uniquely Identifies each record in a table.
    \end{itemize}
    \item Foriegn Key
    \begin{itemize}
        \item Establishes a relationship between two tables
    \end{itemize}
    \item One to One
    \begin{itemize}
        \item Each record in entity A is related to exactly one record in Entity B and vice versa
    \end{itemize}
    \item One to Many
    \begin{itemize}
        \item Each record in entity A can relate to many records in Entity B, but each record in Entity B is related to only one record in Entity A
    \end{itemize}
    \item Many to Many
    \begin{itemize}
        \item Records in Entity A can relate to many records in Entity B, and vice versa
    \end{itemize}
    \end{itemize}
\end{itemize}

\begin{table}
\centering
\begin{tabular}{l|r}
Symbol/Keys & Description \\\hline
PK & PrimaryKey \\
FK & ForiegnKey\\
---\textbar{}------\textbar{}--- & One to One\\
---\textbar{}------\textless{}--- & One to Many\\
-\textgreater{}------\textless{}- & Many to Many
\end{tabular}
\caption{\label{tab:widgets}Symbols/Keys and Description}
\end{table}
\subsection{Definitions}
\begin{itemize}
\item Passenger:
    \begin{itemize}
    \item PK    PassengerID
    \item       FirstName
    \item       LastName
    \item       Email
    \item       Phone
    \item       PassportNumber
    \end{itemize}
\item Ticket:
    \begin{itemize}
    \item PK    TicketID
    \item       SeatNumber
    \item       Class
    \item       Price
    \item FK    BookingID
    \item FK    FlightID
    \end{itemize}
\item Booking:
    \begin{itemize}
    \item PK    BookingID
    \item       BookingDate
    \item       TotalAmount
    \item       PaymentStatus
    \item FK    PassengerID
    \end{itemize}
\item Payment:
    \begin{itemize}
    \item PK    PaymentID
    \item       Amount
    \item       Method
    \item FK    BookingID
    \end{itemize}
\item Flight:
    \begin{itemize}
    \item PK    FlightNumber
    \item       DepartureTime
    \item       ArrivalTime
    \item       Status
    \item FK    AircraftID
    \item FK    DepartureAirport
    \end{itemize}
\item Aircraft:
    \begin{itemize}
    \item PK    AircraftID
    \item       Model
    \item       Manufacturer
    \item       Capacity
    \end{itemize}
\item Airport:
    \begin{itemize}
    \item PK    AirportID
    \item       Name
    \item       City
    \item       Country
    \item       Code
    \end{itemize}
\item Employee:
    \begin{itemize}
    \item PK    EmployeeID
    \item       Name
    \item       Position
    \item       Email
    \item FK    AssignedFlightID
    \end{itemize}
\end{itemize}

\section{Level 1 Data Flow Diagram}


\subsection{DFD Symbols}
\begin{figure}
    \centering
    \includegraphics[width=0.25\linewidth]{DFD-Basic-Symbols.jpg}
    
\end{figure}

\subsection{Definitions}

\begin{itemize}
\item Subsystem:
    \begin{itemize}
    \item Airline Flight Management
    \item User Management
    \item Booking Subsystem
    \item Payment Processing
    \end{itemize}
\item Entity:
    \begin{itemize}
    \item Admin
    \item Customer
    \end{itemize}
\item Data Storage:
    \begin{itemize}
    \item User Database
    \item Flight Database
    \item Payment Records
    \end{itemize}
\end{itemize}

\subsection{Level 1 DFD Summary}

This Data Flow Diagram correctly follows standard conventions, ensuring that all data movement is clearly mapped, and all processes and storage points are logically connected. The interactions between the Customer/Admin and the internal subsystems are carefully defined to reflect real-world airline operations, providing a clear visualization of the system's functional structure.

Within this diagram, we can see a well-structured breakdown of the airline booking system into four core subsystems, two external entities, and three primary data stores, all interconnected through clearly labeled data flows. These components together enable seamless airline operations including user management, flight handling, booking, and payment processing.

\section{Level 2 Data Flow Diagram}

\subsection{Definitions}

\section{Use Case Diagram}

\subsection{Definitions}

\bibliographystyle{alpha}
\bibliography{sample}

\end{document}
