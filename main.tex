\documentclass{article}

\usepackage[english]{babel}

\usepackage[letterpaper,top=1in,bottom=1in,left=1in,right=1in,marginparwidth=1in]{geometry}

% Useful packages
\usepackage{amsmath}
\usepackage{graphicx}
\usepackage[colorlinks=true, allcolors=blue]{hyperref}

\title{Airline Booking System}
\author{Jean-Denis de Beauvoir, Miguel Canelo, Brian Ortiz}

\begin{document}
\maketitle

\begin{abstract}
The airline booking system allows customers to efficiently manage and check flights and their flight tickets and account data. 
The Airline Management System enables efficient handling of passenger, flight, and booking data while ensuring smooth operational throughput even during periods of high demand and numerous concurrent transactions. In addition to managing core functionalities such as flight scheduling, ticketing, and payment processing, the system also supports report generation to aid in various airline operations and decision-making processes. The system integrates a variety of automated modules to streamline workflows across departments, including reservations, customer service, and flight operations. To illustrate its functionality and design, we developed multiple visual models including context flow diagrams, data flow diagrams(Levels 1 and 2), class diagrams, entity-relationship diagrams, and use case diagrams. These models will provide a comprehensive overview of the system's architecture and behavior.
\end{abstract}

\section{Introduction}

Add Text

\section{Context Flow Diagram}

\subsection{Definitions}

\begin{itemize}
\item Employee:
    \begin{itemize}
    \item Those who work for the Airline.
    \end{itemize}
\item Administration:
    \begin{itemize}
    \item Those who run the Airline Management System.
    \end{itemize}
\item Customer:
    \begin{itemize}
    \item The User/Person who will be buying tickets/flights.
    \end{itemize}
\item Aircraft:
    \begin{itemize}
    \item The Plane used to transport different flights/Passengers.
    \end{itemize}
\end{itemize}

\subsection{Symbols}
\begin{itemize}
\item Squares:
    \begin{itemize}
        \item The different squares represent external entities in the context flow diagram.
    \end{itemize}
\end{itemize}

\begin{itemize}
\item Arrows:
    \begin{itemize}
        \item Represent different processes relating entities together and describe what they may do while interacting.
    \end{itemize}
\end{itemize}

\begin{itemize}
\item Circle:
    \begin{itemize}
        \item Represents the Central Entity that all the other external entities interact with.
    \end{itemize}
\end{itemize}

\section{Entity Relationship Diagram}
\subsection{Diagram Symbols}
\begin{itemize}
\item Key Descriptions:
    \begin{itemize}
    \item Primary Key
    \begin{itemize}
        \item Uniquely Identifies each record in a table.
    \end{itemize}
    \item Foriegn Key
    \begin{itemize}
        \item Establishes a relationship between two tables
    \end{itemize}
    \item One to One
    \begin{itemize}
        \item Each record in entity A is related to exactly one record in Entity B and vice versa
    \end{itemize}
    \item One to Many
    \begin{itemize}
        \item Each record in entity A can relate to many records in Entity B, but each record in Entity B is related to only one record in Entity A
    \end{itemize}
    \item Many to Many
    \begin{itemize}
        \item Records in Entity A can relate to many records in Entity B, and vice versa
    \end{itemize}

    \end{itemize}
\begin{table}
\centering
\begin{tabular}{l|r}
Symbol/Keys & Description \\\hline
PK & PrimaryKey \\
FK & ForiegnKey\\
---\textbar{}------\textbar{}--- & One to One\\
---\textbar{}------\textless{}--- & One to Many\\
-\textgreater{}------\textless{}- & Many to Many
\end{tabular}
\caption{\label{tab:widgets}Symbols/Keys and Description}
\end{table}
\subsection{Definitions}
\begin{itemize}
\item Passenger:
    \begin{itemize}
    \item PK    PassengerID
    \item       FirstName
    \item       LastName
    \item       Email
    \item       Phone
    \item       PassportNumber
    \end{itemize}
\item Ticket:
    \begin{itemize}
    \item PK    TicketID
    \item       SeatNumber
    \item       Class
    \item       Price
    \item FK    BookingID
    \item FK    FlightID
    \end{itemize}
\item Booking:
    \begin{itemize}
    \item PK    BookingID
    \item       BookingDate
    \item       TotalAmount
    \item       PaymentStatus
    \item FK    PassengerID
    \end{itemize}
\item Payment:
    \begin{itemize}
    \item PK    PaymentID
    \item       Amount
    \item       Method
    \item FK    BookingID
    \end{itemize}
\item Flight:
    \begin{itemize}
    \item PK    FlightNumber
    \item       DepartureTime
    \item       ArrivalTime
    \item       Status
    \item FK    AircraftID
    \item FK    DepartureAirport
    \end{itemize}
\item Aircraft:
    \begin{itemize}
    \item PK    AircraftID
    \item       Model
    \item       Manufacturer
    \item       Capacity
    \end{itemize}
\item Airport:
    \begin{itemize}
    \item PK    AirportID
    \item       Name
    \item       City
    \item       Country
    \item       Code
    \end{itemize}
\item Employee:
    \begin{itemize}
    \item PK    EmployeeID
    \item       Name
    \item       Position
    \item       Email
    \item FK    AssignedFlightID
    \end{itemize}
\end{itemize}

\section{Data Flow Diagram}
\subsection{Data Flow Diagram}

Use the table and tabular environments for basic tables --- see Table~\ref{tab:widgets}, for example. For more information, see this help article on \href{https://www.overleaf.com/learn/latex/tables}{tables}. 

\subsection{subsection TEXT}

Comments can be added to your project by highlighting some text and clicking ``Add comment'' in the top right of the editor pane. To view existing comments, click on the Review menu in the toolbar above. To respond to a comment, click on the Reply button in the lower right corner of the comment. You can close the Review pane by clicking its name on the toolbar when you are done reviewing for the time being.

Track changes are available in all of our \href{https://www.overleaf.com/user/subscription/plans}{premium plans} and can be toggled on or off using the option at the top of the Review pane. Track changes allow you to keep track of every change made to the document along with the person making the change. 

\subsection{How to add Lists}

You can make lists with automatic numbering \dots

\begin{enumerate}
\item Like this,
\item and like this.
\end{enumerate}
\dots or bullet points \dots
\begin{itemize}
\item Like this,
\item and like this.
\end{itemize}

\subsection{How to write Mathematics}

\LaTeX{} is great at typesetting mathematics. Let $X_1, X_2, \ldots, X_n$ be a sequence of independent and identically distributed random variables with $\text{E}[X_i] = \mu$ and $\text{Var}[X_i] = \sigma^2 < \infty$, and let
\[S_n = \frac{X_1 + X_2 + \cdots + X_n}{n}
      = \frac{1}{n}\sum_{i}^{n} X_i\]
denote their mean. Then as $n$ approaches infinity, the random variables $\sqrt{n}(S_n - \mu)$ converge in distribution to a normal $\mathcal{N}(0, \sigma^2)$.


\subsection{How to change the margins and paper size}

Usually the template you're using will have the page margins and paper size set correctly for that use-case. For example, if you're using a journal article template provided by the journal publisher, that template will be formatted according to their requirements. In these cases, it's best not to alter the margins directly.

If however you're using a more general template, such as this one, and would like to alter the margins, a common way to do so is via the geometry package. You can find the geometry package loaded in the preamble at the top of this example file, and if you'd like to learn more about how to adjust the settings, please visit this help article on \href{https://www.overleaf.com/learn/latex/page_size_and_margins}{page size and margins}.

\subsection{How to change the document language and spell check settings}

Overleaf supports many different languages, including multiple different languages within one document. 

To configure the document language, simply edit the option provided to the babel package in the preamble at the top of this example project. To learn more about the different options, please visit this help article on \href{https://www.overleaf.com/learn/latex/International_language_support}{international language support}.

To change the spell check language, simply open the Overleaf menu at the top left of the editor window, scroll down to the spell check setting, and adjust accordingly.

\subsection{How to add Citations and a References List}

You can simply upload a \verb|.bib| file containing your BibTeX entries, created with a tool such as JabRef. You can then cite entries from it, like this: \cite{greenwade93}. Just remember to specify a bibliography style, as well as the filename of the \verb|.bib|. You can find a \href{https://www.overleaf.com/help/97-how-to-include-a-bibliography-using-bibtex}{video tutorial here} to learn more about BibTeX.

If you have an \href{https://www.overleaf.com/user/subscription/plans}{upgraded account}, you can also import your Mendeley or Zotero library directly as a \verb|.bib| file, via the upload menu in the file-tree.

\subsection{Good luck!}

We hope you find Overleaf useful, and do take a look at our \href{https://www.overleaf.com/learn}{help library} for more tutorials and user guides! Please also let us know if you have any feedback using the Contact Us link at the bottom of the Overleaf menu --- or use the contact form at \url{https://www.overleaf.com/contact}.

\bibliographystyle{alpha}
\bibliography{sample}

\end{document}
